\maketitle

\section{Laboratorio}
Realizar el algoritmo del codificador y decodificador para la codificación Huffman mediante el software de Matlab.

\section{Solución}
\subsection{Codificador}

\lstset{language=Matlab,%
    %basicstyle=\color{red},
    breaklines=true,%
    morekeywords={matlab2tikz},
    keywordstyle=\color{blue},%
    morekeywords=[2]{1}, keywordstyle=[2]{\color{black}},
    identifierstyle=\color{black},%
    stringstyle=\color{mylilas},
    commentstyle=\color{mygreen},%
    showstringspaces=false,%without this there will be a symbol in the places where there is a space
    numbers=left,%
    numberstyle={\tiny \color{black}},% size of the numbers
    numbersep=9pt, % this defines how far the numbers are from the text
    emph=[1]{for,end,break},emphstyle=[1]\color{red}, %some words to emphasise
    %emph=[2]{word1,word2}, emphstyle=[2]{style},    
}

Inicialmente, se implementa el siguiente codigo.

\begin{lstlisting}[language=Matlab,breaklines = true,frame      = false]
   mensaje = fopen(in); % file open
   lectura = fscanf(mensaje,'%c'); % Se lee la data del archivo
   tamanioLectura = length(lectura); % Cantidad de caracteres del archivo
   toAscii = double(lectura); % Se convierten las letras a numeros por ascii
   inAscii = 32; % espacio
   endAscii = 126; % ~
   
\end{lstlisting}

Donde $in$, es un archivo de tipo .txt que contiene un mensaje en texto. Se lee el contenido del archivo y se convierten las palabras a el código Ascii. Se analizaran los caracteres del código Ascii imprimibles, los cuales van desde 'espacio' = 32 hasta 126 = \~  .

\begin{lstlisting}[language=Matlab,breaklines = true,frame      = false]
   proLetras = []; % Probabilidad de cada letra
   
   % Calculo de las probabilidades
   for i=inAscii:endAscii % Desde el espacio hasta la z
       proLetras = [proLetras,length(find(toAscii==i))/tamanioLectura]; % Se aniade su probabilidad
   end
   
\end{lstlisting}

Luego se calcula la probabilidad de que salga un carácter en el texto, para lo cual se cuentan la cantidad de veces que aparece cada carácter y se divide por la cantidad de carácteres.

\begin{lstlisting}[language=Matlab,breaklines = true,frame      = false]
   [dic,M] = Huffman_Dict(proLetras);% Diccionario
   ldic = length(dic);
   dic2 = [];
   a = '';

   % Se convierten los datos del diccionario a texto
    for i=1:ldic    
        for j=1:length(dic(i).a)        
            a = strcat(a,string(dic(i).a(j)));
        end
        dic2=[dic2,string(a)];
        a='';
    end
   
\end{lstlisting}

Ahora, en base a las probabilidades calculadas, se usa la función de $Huffman\_Dict()$ entregada por el profesor, para obtener el diccionario, luego se convierten los datos del diccionario a tipo String para facilitar su manejo.

\begin{lstlisting}[language=Matlab,breaklines = true,frame      = false]
    textoCodificado = '';

    % Se busca cada letra del mensaje y se guarda con su valor del
    %diccionario
    for i=1:length(toAscii)
        textoCodificado = strcat(textoCodificado,string(dic2(toAscii(i)-(inAscii-1))));
    end

    % Se convierte de String a vector de char
    codificado = convertStringsToChars(textoCodificado);
    diccionario = dic2;
   
\end{lstlisting}

Finalmente, se recorre el mensaje a transmitir, donde se buscara cada carácter y se busca su representación en el diccionario, para finalmente concatenarlo a la salida del codificador.\\
La salida del codificador sera la señal codificada y el diccionario.

\subsection{Decodificador}

El código del decodificador se muestra a continuación.

\begin{lstlisting}[language=Matlab,breaklines = true,frame      = false]
    function[out] = Huffman_Deco(in,dic)

    % in: entrada de datos binaria
    % dic: diccionario
    
    lin = length(in);
    a = '';
    mensaje = [];
    
    for i=1:lin
        a = strcat(a,in(i)); % Se concatena la codificacion de a un bit
        if find(dic==a) % Se analiza si lo concatenado existe en el diccionario
            mensaje = [mensaje,find(dic==a)+31];
            a='';
        end
    end
    
    out = char(mensaje);
   
\end{lstlisting}

Para el decodificador, simplemente se recorre la señal codificada bit a bit y se va concatenando cada bit a una variable, por cada bit añadido, se busca si lo que se lleva concatenado existe en el diccionario, si existe, se analiza a que carácter pertenece y se concatena en la señal de salida, y se sigue el mismo procedimiento hasta finalizar el texto codificado.

